% Document encoding: ISO 8859-1
\documentclass{standalone}

\usepackage[latin1]{inputenc}

\usepackage{tikz}
\usetikzlibrary{calc}
\usepackage{ifthen}

% CONFIGURE THE BOARD

% strength of the drawn lines in percent (0: white, 50: medium gray, 100: black)
\def\strength{20}

% grid and PCB size
\pgfmathsetmacro{\raster}{2.54mm}
\pgfmathsetmacro{\pcbLength}{160mm}
\pgfmathsetmacro{\pcbWidth}{100mm}

% do you want stripboard or perfboard?
\newboolean{strips}
\setboolean{strips}{true} % set to false for perfboard

% radius of the pads
\def\padRadius{0.5mm}

% don't edit the following unless you know what you're doing
\pgfmathsetmacro{\nx}{floor(\pcbLength/\raster) - 1}
\pgfmathsetmacro{\ny}{floor(\pcbWidth/\raster) - 1}

\pgfmathsetmacro{\startX}{(\pcbLength - \nx*\raster)/2}
\pgfmathsetmacro{\startY}{(\pcbWidth - \ny*\raster)/2}

\begin{document}
  \begin{tikzpicture}
    \begin{scope}[thick, black!\strength]
    \foreach \row in {0,...,\ny}
    {
      \pgfmathsetmacro{\py}{\startY + \row*\raster}
      \pgfmathsetmacro{\endX}{\pcbLength - \startX}
      \ifthenelse{\boolean{strips}}{\draw (\startX pt, \py pt) -- (\endX pt, \py pt);}{}
      \foreach \col in {0,...,\nx}
      {
        \pgfmathsetmacro{\px}{\startX + \col*\raster}
        \fill[white] (\px pt, \py pt) circle(\padRadius);
        \draw (\px pt, \py pt) circle(\padRadius);
      }
    }
    \end{scope}
    \draw (0,0) rectangle (\pcbLength pt,\pcbWidth pt);
  \end{tikzpicture}
\end{document}
